% CHAPTER 1
\chapter{INTRODUCTION}
\label{chp:introduction}

The genetic information of the cell is stored in DNA. DNA contains thousands of genes that each lead to the production of a particular protein. Proteins are building blocks that perform all the necessary functions of the cell. The flow of genetic information from DNA to protein, which is also called the central dogma, involves the transcription of DNA to RNA and the translation of RNA to protein. The amount of expression of the set of genes is critical and determines the state of a cell. As such, gene expression has to be regulated precisely.

Most of the research in the last decade has focused on the first level of gene expression regulation which is the transcriptional control. However, recent studies approved that, in addition to the transcriptional steps, post-transcriptional steps also have significant impact in regulation of gene expression. Post-transcriptional regulation (PTR) is a complex system that controls every aspect of RNA metabolism including splicing, stability, localization, and degradation. PTR is mediated by interactions between cis-regulatory elements in mRNA and trans-acting factors such as RNA-binding proteins (RBPs) and microRNAs (miRNA). Understanding these interactions is crucial because dysregulation in PTR networks are associated with several diseases including neurodegenerative disorders and various types of cancer.

RBPs bind to target mRNAs and control their fates in every step from splicing to translation.  RBPs recognize short sequence and/or structure motifs in their target mRNAs. Target sites of majority of RBPs remain uncharacterized. However, recently several high-throughput experimental techniques have been developed to characterize the \textit{in vitro} and \textit{in vivo} binding specificities of RBPs. 

MiRNAs form another class of post-transcriptional regulators that are known to downregulate the gene expression by transcript degradation or translation inhibition. MiRNAs are non-coding RNAs of length 21-24 nts which interact with their target mRNA by complementary base pairing with their seed region. In addition to the limited number of experimentally verified targets, several computational methods have been developed to predict target sites of miRNAs. 

Despite the fact that miRNAs and RBPs regulate the expression of overlapping sets of mRNAs , systematic studies that consider the effects of both of these factors are lacking. Additionally, recent studies show that RBPs and miRNAs are involved in cooperative and competitive interactions that lead to an even more complex gene regulation network. In this thesis, our aim is to better understand PTR by modeling the combined effect of RBPs and miRNAs and the cooperative and competitive interactions between them.

This thesis is organized into five chapters as follows:
\begin{itemize}

\item \textbf{Chapter 2} provides background about RBPs, miRNAs, and discusses the experimental and computational approaches to identify their binding sites. This chapter also reviews previous work that consider the effect of both RBPs and miRNAs.

\item \textbf{Chapter 3} describes the methods and datasets used in this thesis. In particular, we explain how we mapped the genome-wide binding sites of RBPs and miRNAs, and predicted RNA secondary structure.  We also describe the details of the statistical model that we developed to predict mRNA stability.

\item \textbf{Chapter 4} explains our results that support the need for modeling RBPs and miRNAs together to model PTR.

\item \textbf{Chapter 5} summarizes the results and discusses possible future directions of this work.
\end{itemize}