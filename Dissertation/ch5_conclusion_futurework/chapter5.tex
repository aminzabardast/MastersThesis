% CHAPTER 5
\chapter{CONCLUSION AND FUTURE WORK}
\label{chp:chapter5}

\section{Conclusion}

PTR is mediated by the interactions of RBPs and miRNAs with target sites on mRNAs. Recent studies show that these two major post-transcriptional regulators do not act independently. Indeed, they are involved in competitive or cooperative interactions with each other. In this thesis, we considered the effect of multiple factors concurrently.  First, we mapped all RBP and miRNA binding sites on human 3’UTRs. We took advantage of \textit{in vitro} (RNAcompete), \textit{in vivo} (CLIP, gPARCLIP) and computational methods (TargetScan, PicTar) in order to identify and predict binding sites of RBPs and miRNAs. We also considered the secondary structure of 3’UTRs by predicting with computational methods RNAplfold and Sfold.   

First, we compared CLIP-supported sites and other sites in terms of accessibility and conservation. We repeated this analysis considering CLIP-supported sites of PUM2, HuR, IGF2BP1-3, and QKI RBPs. We observed that CLIP-supported sites of these RBPs are more conserved compared to other sites. In case of accessibility, we observed that CLIP-supported sites of HuR are more accessible which is in line with previous studies \cite{hur_accessibility, kazan_10}. It also holds true for CLIP-supported sites of QKI. However, in contrast to what we expected, CLIP-supported sites of PUM2 are less accessible. Besides, CLIP-supported sites of IGF2BP1-3 are also less accessible compared to other sites.

Next, we utilized the HuR knockdown dataset and checked whether CLIP-supported sites of HuR show any difference compared to other sites in terms of functional outcome upon HuR depletion. Indeed, we observed that transcripts with CLIP-supported HuR sites are destabilized more compared to transcripts with no CLIP-supported sites at all. We also identified HuR sites that overlap with binding sites of other factors and considered them as being in competition. Next, we investigated the effect of competition to HuR function. We observed that transcripts in which all HuR sites were in competition with other factors show less change upon HuR knockdown. 

In order to find potential interacting factors, for each pair of factors, we checked whether their binding sites co-occur more often than expected by random chance. We observed that binding sites of miRNAs tend to occur near PUM2 sites. In order to investigate the possible interaction between these factors, we classified transcripts into two groups: first, those transcripts that contain a site of PUM2 and a site of one of its interacting miRNAs in a stem-loop. Second, those transcripts that still contain sites of both of these factors but not in a stem-loop. We observed that transcripts in the first group have shorter half-lives compared to the second one.

Besides, we observed that there are no or little amount of interacting factors in the 3’-end of binding sites of PUM1(2). We think that it is because PUM1(2) tends to bind to the 3’-end of transcripts.  We divided the transcripts which contain binding sites of PUM1(2) into several windows and counted the number of binding sites in each of them. As expected, we observed that the nearest window to the 3'-end of transcripts contains the most binding sites of PUM1(2). Furthermore, we assessed the effect of number of PUM1(2) binding sites on mRNA stability and observed that transcripts with more PUM1(2) binding sites tend to have shorter half-lives. Besides, we observed that binding sites of PUM1(2) tend to col-localize near each other. We supposed that in such conditions, PUM1(2) may interact with itself. We classified the transcripts into two groups: (i) Transcripts which contain binding sites of PUM1(2) with at least two sites close to each other. (ii) Transcripts which contain PUM1(2) sites but there are no binding sites close to each other. We observed that transcripts in the first group have shorter half-lives. This indicates that binding sites of PUM1(2) may involve in interaction with each other which may lead to faster degradation of their target mRNA.

Finally, we trained a logistic regression with features compiled from counts of sites of miRNAs, RBPs and dinucleotides to predict half-life and mRNA abundance. The high performance of the dinucleotide-only model is in line with 2013 DREAM5 challenge results where motif models that consider dinucleotide content perform well \cite{weirauch_13}. We hypothesize that dinucleotide features capture the binding motifs of RBPs or miRNAs. However, we observed that including counts of RBP and miRNAs sites improves the performance, and we achieved a 10-fold CV AU-ROC of 0.86 in predicting half-life in BEAS-2B cells. We also considered alternative ways to compile the features: (i) counting RBP sites that are only gPARCLIP or CLIP-supported; (ii) summing the accessibility of sites; and (iii) considering the competition between the factors. However, these modifications did not increase the predictive performance. The first modification may not have improved performance as only 7 out of 19 RBPs that are in activators and repressors group have CLIP data. Also, counting sites with gPARCLIP support may not result in an accurate estimate of the activity of a specific RBP as the identity of the bound RBP in a gPARCLIP-determined peak is unknown. Lastly, gPARCLIP and most CLIP experiments have been performed in HEK293 cells, whereas we predict half-life and transcript abundance in other cell types such as BEAS-2B and MCF7 cells. The second modification where we considered the accessibility of sites might result in a better performance as our knowledge of mRNA secondary structure gets more accurate. Another issue related to this modification could be our assumption that all RBPs and miRNAs prefer binding to accessible regions. In reality, most of the RBPs have unknown secondary structure preferences and increased knowledge on this would be instrumental.

\section{Future directions}

Despite recent efforts in identifying cooperative and competitive interactions between RBPs and miRNAs, a number of challenges remain in modeling the interaction between these PTR factors. Here we mention some of them which can be handled in the future.

In this thesis, we limited most of our analysis to RBPs HuR and PUM1(2). That’s because they are well-characterized and their effect on mRNA stability is known. With the recent advances in this field, more effect of factors becomes known to scientists. As a future work, we can investigate competitive and cooperative interactions between wider range of factors.

Furthermore, there are also limited datasets that measure genome-wide effect of factors upon their depletion or transfection. We used HuR knockdown dataset to investigate the effect of competition to HuR function. Although we repeated this analysis using for example QKI knockdown dataset, but as the effect of QKI on mRNA stability is not very well known, we cannot reason based on them. In the future, we can repeat our analyses using knockdown or transfection dataset of other RBPs.

Experimental techniques can query mRNA secondary structure \textit{in vivo} \cite{rouskin_14, wan_14}; however, the coverage of the resulting profiles are limited. Recently developed icSHAPE technique \cite{spitale_15} is shown to query the secondary structure of all four bases in mouse ES cells. icSHAPE secondary structure profiles can be utilized in our framework as they become available for human cell types. 